\batchmode
\makeatletter
\def\input@path{{/Users/sergei.winitzki/Code/talks/ftt-fp/}}
\makeatother
\documentclass[english]{beamer}
\usepackage[T1]{fontenc}
\usepackage[latin9]{inputenc}
\setcounter{secnumdepth}{3}
\setcounter{tocdepth}{3}
\usepackage{babel}
\usepackage{amstext}
\usepackage{wasysym}
\usepackage[all]{xy}
\ifx\hypersetup\undefined
  \AtBeginDocument{%
    \hypersetup{unicode=true,pdfusetitle,
 bookmarks=true,bookmarksnumbered=false,bookmarksopen=false,
 breaklinks=false,pdfborder={0 0 1},backref=false,colorlinks=true}
  }
\else
  \hypersetup{unicode=true,pdfusetitle,
 bookmarks=true,bookmarksnumbered=false,bookmarksopen=false,
 breaklinks=false,pdfborder={0 0 1},backref=false,colorlinks=true}
\fi

\makeatletter
%%%%%%%%%%%%%%%%%%%%%%%%%%%%%% Textclass specific LaTeX commands.
% this default might be overridden by plain title style
\newcommand\makebeamertitle{\frame{\maketitle}}%
% (ERT) argument for the TOC
\AtBeginDocument{%
  \let\origtableofcontents=\tableofcontents
  \def\tableofcontents{\@ifnextchar[{\origtableofcontents}{\gobbletableofcontents}}
  \def\gobbletableofcontents#1{\origtableofcontents}
}
\newenvironment{lyxcode}
  {\par\begin{list}{}{
    \setlength{\rightmargin}{\leftmargin}
    \setlength{\listparindent}{0pt}% needed for AMS classes
    \raggedright
    \setlength{\itemsep}{0pt}
    \setlength{\parsep}{0pt}
    \normalfont\ttfamily}%
   \def\{{\char`\{}
   \def\}{\char`\}}
   \def\textasciitilde{\char`\~}
   \item[]}
  {\end{list}}

%%%%%%%%%%%%%%%%%%%%%%%%%%%%%% User specified LaTeX commands.
\usetheme[secheader]{Boadilla}
\usecolortheme{seahorse}
\title[Chapter 11: Monad transformers]{Chapter 11:
Computations in a functor context III}
\subtitle{Monad transformers}
\author{Sergei Winitzki}
\date{2019-01-05}
\institute[ABTB]{Academy by the Bay}
\setbeamertemplate{headline}{} % disable headline at top
\setbeamertemplate{navigation symbols}{} % disable navigation bar at bottom
\usepackage[all]{xy} % xypic
\usepackage[nocenter]{qtree} % simple tree drawing
\usepackage{relsize} % make math symbols larger or smaller
\newcommand{\bef}{\ensuremath\raisebox{2pt}{$\mathsmaller{\mathsmaller{\circ}}$}\hspace{-3.3pt},}
%\makeatletter
% Macros to assist LyX with XYpic when using scaling.
\newcommand{\xyScaleX}[1]{%
\makeatletter
\xydef@\xymatrixcolsep@{#1}
\makeatother
} % end of \xyScaleX
\makeatletter
\newcommand{\xyScaleY}[1]{%
\makeatletter
\xydef@\xymatrixrowsep@{#1}
\makeatother
} % end of \xyScaleY

\makeatother

\begin{document}
\frame{\titlepage}
\begin{frame}{Computations within a functor context: Combining monads}

Programs often need to combine monadic effects
\begin{itemize}
\item ``Effect'' $\equiv$ what else happens in {\footnotesize{}$A\Rightarrow M^{B}$}
besides computing $B$ from $A$
\item Examples of effects for some standard monads:
\begin{itemize}
\item \texttt{\textcolor{blue}{\footnotesize{}Option}} -- computation will
have no result or a single result
\item \texttt{\textcolor{blue}{\footnotesize{}List}} -- computation will
have zero, one, or multiple results
\item \texttt{\textcolor{blue}{\footnotesize{}Either}} -- computation may
fail to obtain its result, reports error
\item \texttt{\textcolor{blue}{\footnotesize{}Reader}} -- computation needs
to read an external context value
\item \texttt{\textcolor{blue}{\footnotesize{}Writer}} -- some value will
be appended to a (monoidal) accumulator
\item \texttt{\textcolor{blue}{\footnotesize{}Future}} -- computation will
be scheduled to run later
\end{itemize}
\item How to combine several effects in the same functor block (\texttt{\textcolor{blue}{\footnotesize{}for}}/\texttt{\textcolor{blue}{\footnotesize{}yield}})?
\end{itemize}
{\footnotesize{}\vspace{-0.35cm}}\texttt{\textcolor{blue}{\footnotesize{}}}%
\begin{minipage}[t]{0.49\columnwidth}%
\begin{lyxcode}
\textcolor{darkgray}{\footnotesize{}//~This~is~not~valid~Scala!}{\footnotesize\par}

\textcolor{blue}{\footnotesize{}val~result~=~for~\{~i~$\leftarrow$~1~to~n}{\footnotesize\par}

\textcolor{blue}{\footnotesize{}~~~j~$\leftarrow$~Future~\{~q(i)~\}}{\footnotesize\par}

\textcolor{blue}{\footnotesize{}~~~k~$\leftarrow$~maybeError(j)~:~Try{[}Int{]}}{\footnotesize\par}

\textcolor{blue}{\footnotesize{}\}~yield~f(k)}{\footnotesize\par}

\textcolor{darkgray}{\footnotesize{}//~What~should~be~the~type~of~result??}{\footnotesize\par}
\end{lyxcode}
%
\end{minipage}\texttt{\textcolor{blue}{\footnotesize{}~ ~ ~}}%
\begin{minipage}[t]{0.49\columnwidth}%
\begin{lyxcode}
\textcolor{blue}{\footnotesize{}~~}\textcolor{darkgray}{\footnotesize{}//~This~is~not~valid~Scala!}{\footnotesize\par}

\textcolor{blue}{\footnotesize{}(1~to~n).flatMap~\{~i~$\Rightarrow$}{\footnotesize\par}

\textcolor{blue}{\footnotesize{}~~~Future(q(i)).flatMap~\{~j~$\Rightarrow$}{\footnotesize\par}

\textcolor{blue}{\footnotesize{}~~~~~maybeError(j).map~\{~k~$\Rightarrow$}{\footnotesize\par}

\textcolor{blue}{\footnotesize{}~~~~~~~f(k)}{\footnotesize\par}

\textcolor{blue}{\footnotesize{}~~~~~~~~~\}\}\}}{\footnotesize\par}
\end{lyxcode}
%
\end{minipage}\texttt{\textcolor{blue}{\footnotesize{}\medskip{}
}}{\footnotesize\par}
\begin{itemize}
\item The code will work if we ``unify'' all effects in a new, larger
monad
\item Need to compute the type of new monad that contains all given effects
\end{itemize}
\end{frame}

\begin{frame}{Combining monadic effects I. Trial and error}

There are several ways of combining two monads into a new monad:
\begin{itemize}
\item If $M_{1}^{A}$ and $M_{2}^{A}$ are monads then $M_{1}^{A}\times M_{2}^{A}$
is also a monad
\begin{itemize}
\item But $M_{1}^{A}\times M_{2}^{A}$ describes two separate values with
two separate effects
\end{itemize}
\item If $M_{1}^{A}$ and $M_{2}^{A}$ are monads then $M_{1}^{A}+M_{2}^{A}$
is usually not a monad
\begin{itemize}
\item If it worked, it would be a choice between two different values /
effects
\end{itemize}
\item If $M_{1}^{A}$ and $M_{2}^{A}$ are monads then one of $M_{1}^{M_{2}^{A}}$
or $M_{2}^{M_{1}^{A}}$ is often a monad
\item Examples and counterexamples for functor composition:
\begin{itemize}
\item Combine (p).flatMap(q andThen lift$_{1}$) = lift$_{1}$(p flatMap
q)}}{\footnotesize\par}
\begin{itemize}
\item Rewritten equivalently through $\text{flm}_{M}:\left(A\Rightarrow M^{B}\right)\Rightarrow M^{A}\Rightarrow M^{B}$
as
\end{itemize}
\begin{center}
{\footnotesize{}\vspace{-0.2cm}\hspace{-0.0cm}$\text{lift}_{1}\bef\text{flm}_{\text{BigM}}\left(q\bef\text{lift}_{1}\right)=\text{flm}_{M_{1}}q\bef\text{lift}_{1}$}{\footnotesize\par}
\par\end{center}
\begin{itemize}
\item {\footnotesize{}\vspace{-0.2cm}\hspace{-0.0cm}}Rewritten in terms
of Kleisli composition:
\end{itemize}
\begin{center}
{\footnotesize{}\vspace{-0.2cm}\hspace{-0.0cm}$\big(b^{:X\Rightarrow M_{1}^{Y}}\bef\text{lift}_{1}\big)\diamond\big(c^{:Y\Rightarrow M_{1}^{Z}}\bef\text{lift}_{1}\big)=\left(b\diamond c\right)\bef\text{lift}_{1}$}{\footnotesize\par}
\par\end{center}
\begin{itemize}
\item {\footnotesize{}\vspace{-0.3cm}\hspace{-0.0cm}}Liftings $\text{lift}_{1}$
and $\text{lift}_{2}$ must obey an identity law and a composition
law
\item The laws say that the liftings \textbf{commute with} the monads' operations
\end{itemize}
\end{frame}

\begin{frame}{Laws for monad liftings III. The naturality law}

Show that $\text{lift}_{1}:M_{1}^{A}\Rightarrow\text{BigM}^{A}$ is
a natural transformation 
\begin{itemize}
\item It maps $\text{pure}_{M_{1}}$ to $\text{pure}_{\text{BigM}}$ and
$\text{flm}_{M_{1}}$ to $\text{flm}_{\text{BigM}}$
\begin{itemize}
\item $\text{lift}_{1}$ is a \textbf{monadic morphism} between monads $M_{1}^{\bullet}$
and $\text{BigM}^{\bullet}$
\end{itemize}
\end{itemize}
The (functor) naturality law: 
\[
\text{lift}_{1}\bef\text{fmap}_{B}f^{:X\Rightarrow Y}=\text{fmap}_{M_{1}}f^{:X\Rightarrow Y}\bef\text{lift}_{1}
\]
{\footnotesize{}\vspace{-0.5cm}
\[
\xymatrix{\xyScaleY{2pc}\xyScaleX{3pc}M_{1}^{X}\ar[d]\sb(0.45){\text{fmap}_{M_{1}}\,f^{:X\Rightarrow Y}}\ar[r]\sp(0.45){\ \text{lift}_{1}} & \text{BigM}^{X}\ar[d]\sp(0.45){\text{fmap}_{\text{BigM}}\,f^{:X\Rightarrow Y}}\\
M_{1}^{Y}\ar[r]\sp(0.45){\text{lift}_{1}} & \text{BigM}^{Y}
}
\]
}Derivation of the naturality law:
\begin{itemize}
\item Express $\text{fmap}$ as $\text{fmap}_{M}f=\text{flm}_{M}\left(f\bef\text{pure}_{M}\right)$
for both monads
\item Given $f^{:X\Rightarrow Y}$, use the law {\footnotesize{}$\text{flm}_{M_{1}}q\bef\text{lift}_{1}=\text{lift}_{1}\bef\text{flm}_{\text{BigM}}\left(q\bef\text{lift}_{1}\right)$}
to compute {\footnotesize{}$\text{flm}_{M_{1}}\left(f\bef\text{pure}_{M_{1}}\right)\bef\text{lift}_{1}=\text{lift}_{1}\bef\text{flm}\left(f\bef\text{pure}_{M_{1}}\bef\text{lift}_{1}\right)=\text{lift}_{1}\bef\text{flm}\left(f\bef\text{pure}_{\text{BigM}}\right)=\text{lift}_{1}\bef\text{fmap}_{\text{BigM}}f$}{\footnotesize\par}
\end{itemize}
A monadic morphism is always also a natural transformation of the
functors
\end{frame}

\begin{frame}{Monad transformers I}

\begin{itemize}
\item {\footnotesize{}\vspace{-0.2cm}}Combine $Z\Rightarrow A$ and $1+A$:
only $Z\Rightarrow1+A$ works, not $1+\left(Z\Rightarrow A\right)$
\begin{itemize}
\item It is not possible to combine monads via a natural bifunctor $B^{M_{1},M_{2}}$
\item It is not possible to combine arbitrary monads as $M_{1}^{M_{2}^{\bullet}}$
or $M_{2}^{M_{1}^{\bullet}}$
\end{itemize}
\item The trick: let $M^{\bullet}$ (\textbf{foreign monad}) vary, for a
fixed \textbf{base monad} $L^{\bullet}$
\item The result is a monad transformer $T_{L}^{M,A}$ -- a natural functor
in $M_{2}$
\end{itemize}
A \textbf{monad transformer} for a \textbf{base} monad $L^{\bullet}$
is a type constructor $T_{L}^{M,\bullet}$ parameterized by a monad
$M^{\bullet}$, such that for all monads $M^{\bullet}$
\begin{itemize}
\item $T_{L}^{M,\bullet}$ is a monad (the monad $M$ \textbf{transformed
with} $T_{L}$)
\item ``Lifting'' -- a monadic morphism $\text{lift}_{L}^{M}:M^{A}\leadsto T_{L}^{M,A}$,
natural in $M^{\bullet}$
\item ``Injection'' -- a monadic morphism $\text{inj}:L^{A}\leadsto T_{L}^{M,A}$ 
\item $T_{L}^{M,\bullet}$ is \textbf{monadically natural} in $M^{\bullet}$
\begin{itemize}
\item $T_{L}^{M,\bullet}$ is natural w.r.t.~a monadic functor $M^{\bullet}$
as a type parameter
\item For any monad $N^{\bullet}$ and a monadic morphism $f:M^{\bullet}\leadsto N^{\bullet}$
we need to have a monadic morphism $T_{L}^{M,\bullet}\leadsto T_{L}^{N,\bullet}$
for the transformed monads
\item Naturality will hold if we implement $T_{L}^{M,\bullet}$ only via
$M$'s monad methods 
\item Cf.~\texttt{\textcolor{blue}{\footnotesize{}traverse}}{\small{}$:L^{A}\Rightarrow\left(A\Rightarrow F^{B}\right)\Rightarrow F^{L^{B}}$
-- natural w.r.t.~applicative $F^{\bullet}$}{\small\par}
\end{itemize}
\end{itemize}
\end{frame}

\begin{frame}{Exercises}
\begin{enumerate}
\item {\small{}Show that the method }\texttt{\textcolor{blue}{\footnotesize{}pure}}{\small{}$:A\Rightarrow M^{A}$
is a monadic morphism between monads $\text{Id}^{A}\equiv A$ and
$M^{A}$.}{\small\par}
\item {\small{}Show that $M_{1}^{A}+M_{2}^{A}$ is }\emph{\small{}not}{\small{}
a monad when $M_{1}^{A}\equiv1+A$ and $M_{2}^{A}\equiv Z\Rightarrow A$.}{\small\par}
\end{enumerate}
\end{frame}

\end{document}
